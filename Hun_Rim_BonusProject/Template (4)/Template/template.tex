\documentclass[unicode,11pt,a4paper,oneside,numbers=endperiod,openany]{scrartcl}
\usepackage{amsmath}
\usepackage{listings}
\usepackage{enumitem}

\newcommand{\norm}[1]{\lvert\lvert #1 \rvert\rvert}

\renewcommand{\thesubsection}{\arabic{subsection}}

\input{assignment.sty}
\begin{document}


\setassignment
\setduedate{Wednesday, 22 November 2023, 11:59 PM}

\serieheader{Numerical Computing}{2023}{\textbf{Student:} FULL NAME}{\textbf{Discussed with:} FULL NAME}{Bonus assignment}{}
\newline

\assignmentpolicy


\newpage

\section*{Exercise 1: Inconsistent systems of equations [10 points]}
Consider the following inconsistent systems of equations: \\

\begin{center}
(a) ${A_1x = b_1}$, where

\vspace{10px}

\begin{equation*}
A_1 =
\begin{bmatrix}
1 & 0 \\
1 & 0 \\
1 & 0
\end{bmatrix}
b_1=
\begin{bmatrix}
5 \\
2 \\
4
\end{bmatrix}
\end{equation*}
\end{center}

\begin{center}
(b) ${A_2x = b_2}$, where

\vspace{10px}

\begin{equation*}
A_2 =
\begin{bmatrix}
1 & 1 & 0 \\
0 & 1 & 1 \\
1 & 2 & 1 \\
1 & 0 & 1
\end{bmatrix}
b_2=
\begin{bmatrix}
2 \\
2 \\
3 \\
4
\end{bmatrix}
\end{equation*}
\end{center}

Find the least squares solution ${x^*}$ and compute the Euclidean norm of the residual, SE and RMSE. \\

\textbf{solution:} \\
Least Square solution ${x^*}$ can be obtained by solving the following equation:

\vspace{20px}

\begin{equation}
 A^TAx = A^Tb
\end{equation}

\vspace{20px}

Then from the ${x^*}$ obtained, we can get the residual vector as following and from it, we can calculate the Euclidean norm and proceed to SE (Sum of Square Residuals) and RMSE (Root Mean Squared Error):

\vspace{20px}

\begin{equation}
 r = Ax^* - b
\end{equation}

\vspace{20px}

\begin{equation}
 Euclidean Norm = \norm{r}_2
\end{equation}

\vspace{20px}

\begin{equation}
 SE = \norm{r}_2^2
\end{equation}

\vspace{20px}

\begin{equation}
 RMSE = \sqrt{\frac{SE}{m}}
\end{equation}

\vspace{20px}

Where ${m}$ is the number of rows in residual vector. \\

\newpage
(a)
\begin{equation*}
\begin{bmatrix}
1 & 1 & 1 \\
0 & 0 & 0
\end{bmatrix}
\begin{bmatrix}
1 & 0 \\
1 & 0 \\
1 & 0
\end{bmatrix}
x = \begin{bmatrix}
1 & 1 & 1 \\
0 & 0 & 0
\end{bmatrix}
\begin{bmatrix}
5 \\
2 \\
4
\end{bmatrix}
\end{equation*}
\begin{equation*}
 \begin{bmatrix}
  3 & 0 \\
  0 & 0
 \end{bmatrix}x=
 \begin{bmatrix}
  11 \\
  0
 \end{bmatrix}
\end{equation*}
\begin{equation*}
 x^* = \begin{bmatrix}
        3.6667 \\
        0
       \end{bmatrix}
\end{equation*}
\begin{equation*}
 r^* = \begin{bmatrix}
        1 & 0 \\
        1 & 0 \\
        1 & 0
       \end{bmatrix}
       \begin{bmatrix}
        3.6667 \\
        0
       \end{bmatrix} -
       \begin{bmatrix}
        5 \\
        2 \\
        4
       \end{bmatrix} =
       \begin{bmatrix}
        -1.3333\\
        1.6667 \\
        -0.3333
       \end{bmatrix}
\end{equation*}
\begin{equation*}
 EuclideanNorm = \norm{r}_2 = \sqrt{(-1.3333)^2 + (1.6667)^2 + (-0.3333)^2} = \sqrt{4.6667} \approx 2.1602
\end{equation*}

\begin{equation*}
 SE = \norm{r}_2^2 = (-1.3333)^2 + (1.6667)^2 + (-0.3333)^2 \approx 4.6667
\end{equation*}

\begin{equation*}
 RMSE = \sqrt{\frac{\norm{r}_2^2}{m}} \approx \sqrt{\frac{4.6667}{3}} \approx 1.2472 
\end{equation*}

(b)
\begin{equation*}
\begin{bmatrix}
1 & 0 & 1 & 1 \\
1 & 1 & 2 & 0 \\
0 & 1 & 1 & 1
\end{bmatrix}
\begin{bmatrix}
1 & 1 & 0 \\
0 & 1 & 1 \\
1 & 2 & 1 \\
1 & 0 & 1
\end{bmatrix}
x = \begin{bmatrix}
1 & 0 & 1 & 1 \\
1 & 1 & 2 & 0 \\
0 & 1 & 1 & 1
\end{bmatrix}
\begin{bmatrix}
2 \\
2 \\
3 \\
4
\end{bmatrix}
\end{equation*}

\begin{equation*}
 \begin{bmatrix}
  3 & 3 & 2 \\
  3 & 6 & 3 \\
  2 & 3 & 3
 \end{bmatrix}x = 
 \begin{bmatrix}
  9 \\
  10 \\
  9
 \end{bmatrix}
\end{equation*}

If we re-arrange the formula for ${x^*}$ we get:

\begin{equation}
 x^* = (A^T_2A_2)^{-1}A^T_2b_2
\end{equation}

and the resulting ${x^*}$ will be

\begin{equation*}
 x^* \approx \begin{bmatrix}
        2 \\
        -0.3333 \\
        2
       \end{bmatrix}
\end{equation*}

\begin{equation*}
 r = \begin{bmatrix}
1 & 1 & 0 \\
0 & 1 & 1 \\
1 & 2 & 1 \\
1 & 0 & 1
\end{bmatrix}
\begin{bmatrix}
        2 \\
        -0.3333 \\
        2
\end{bmatrix} -
\begin{bmatrix}
2 \\
2 \\
3 \\
4
\end{bmatrix} \approx
\begin{bmatrix}
-0.3333 \\
-0.3333 \\
0.3333 \\
0
\end{bmatrix}
\end{equation*}

\begin{equation*}
 EuclideanNorm = \norm{r}_2 \approx \sqrt{(-0.3333)^2 + (-0.3333)^2 + (0.3333)^2 + (0)^2} \approx \sqrt{0.3333} \approx 0.5774
\end{equation*}

\begin{equation*}
 SE = \norm{r}_2^2 = (-0.3333)^2 + (-0.3333)^2 + (0.3333)^2 + (0)^2 \approx 0.3333
\end{equation*}

\begin{equation*}
 RMSE = \sqrt{\frac{\norm{r}_2^2}{m}} \approx \sqrt{\frac{0.3333}{4}} \approx 0.2887
\end{equation*}

\section*{Exercise 2: Polynomials models for least squares [20 points]}

\begin{enumerate}[label=(\alph*)]
 \item Write ${leastSquare.m}$ function which calculates least squares, euclidean norm, SE and RMSE of a matrix A and vector b, and write a script ${ex2a.m}$ which computes the result of exercise 1.
\end{enumerate}



\section*{Exercise 3: Analysis of periodic data [20 points]}

\section*{Exercise 4: Data linearization and Levenberg-Marquardt method for the exponential model [20 points]}

\section*{Exercise 5: Tikhonov regularization [15 points]}



\end{document}
